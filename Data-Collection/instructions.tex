\documentclass{article}

\usepackage[LGR, T1]{fontenc} %για γλώσσα
\usepackage[greek,english]{babel}
\usepackage[utf8]{inputenc}

\usepackage{graphicx} % figures
\usepackage{caption}
\usepackage{subcaption}
\usepackage{float}
\usepackage{color}   %May be necessary if you want to color links
\usepackage{textcomp}
\usepackage{xcolor}

\usepackage{hyperref}
\usepackage{tocloft}
\usepackage{fontawesome5}

% Add dots to table of contents
\renewcommand{\cftsecleader}{\cftdotfill{\cftdotsep}}

\hypersetup{
    colorlinks=true,
    linkcolor=black,
    urlcolor=blue,
    citecolor=blue,
}

\begin{document}
\selectlanguage{Greek}
\begin{titlepage}

    \begin{figure}
        \centering
        \includegraphics[scale=0.4]{Upatras}
    \end{figure}

    \begin{center}
        \line(1,0){300}\\
        [0.25in]
        \huge{\bfseries \selectlanguage{greek}Συλλογή δεδομένων για διπλωματική εργασία}\\

        \line(1,0){200}\\
        \begin{figure}[H]
            \centering
            \includegraphics[scale=2.2]{logo}\\
        \end{figure}
    \end{center}

    \begin{flushright}
        \textsc{\\}
        \textsc{\\}

        \textsc{\large \\ Τριανταφυλλόπουλος Παναγιώτης\\ΑΜ 1054367}
        \textsc{\large \\ Τμήμα Μηχ. Η/Υ \& Πληροφορικής}
        \textsc{\large \\ Διπλωματική Εργασία}
    \end{flushright}

\end{titlepage}

\tableofcontents
\thispagestyle{empty}
\cleardoublepage

\setcounter{page}{1}
\selectlanguage{english}
\section*{Disclaimer}
\addcontentsline{toc}{section}{Disclaimer}

\selectlanguage{greek} Αυτό το ερευνητικό έργο θα χρησιμοποιήσει έναν τυχαία δημιουργημένο κωδικό αναγνώρισης για να διασφαλίσει την ανωνυμία όλων των συμμετεχόντων. Δε θα συλλεχθούν ούτε θα αποθηκευθούν προσωπικά στοιχεία ταυτοποίησης και δεν θα γίνει προσπάθεια σύνδεσης οποιουδήποτε συγκεκριμένου συμμετέχοντος με τις απαντήσεις ή τα δεδομένα του. Όλα τα δεδομένα που συλλέγονται κατά τη διάρκεια αυτής της μελέτης θα διατηρηθούν αυστηρά απόρρητα και θα χρησιμοποιηθούν αποκλειστικά για ερευνητικούς σκοπούς. Η συμμετοχή σε αυτή τη μελέτη είναι εντελώς εθελοντική και οι συμμετέχοντες μπορούν να αποσυρθούν ανά πάσα στιγμή χωρίς κυρώσεις ή απώλεια ωφελημάτων που διαφορετικά δικαιούνται.

\section*{Εισαγωγή}
\addcontentsline{toc}{section}{Εισαγωγή}

Καλείστε να συμμετάσχετε στη συλλογή δεδομένων στα πλαίσια της διπλωματικής εργασίας του προπτυχιακού φοιτητή Τριανταφυλλόπουλου Παναγιώτη, η οποία έχει σκοπό τη δημιουργία ενός συστήματος συστάσεων ιστοσελίδων. Για τη συλλογή δεδομένων θα χρειαστείτε έναν φυλλομετρητή (\selectlanguage{english}browser\selectlanguage{greek} - κατά προτίμηση \selectlanguage{english}Google Chrome\selectlanguage{greek} ή \selectlanguage{english}Firefox\selectlanguage{greek}) και να εγκαταστήσετε και να χρησιμοποιήσετε μία επέκταση φυλλομετρητή (\selectlanguage{english}browser extension\selectlanguage{greek}) που έχει δημιουργηθεί από τον φοιτητή.

\section*{Σκοπός συλλογής δεδομένων}
\addcontentsline{toc}{section}{Σκοπός συλλογής δεδομένων}

Ο σκοπός της συλλογής δεδομένων από χρήστες που χρησιμοποιούν την επέκταση φυλλομετρητή είναι η ανάπτυξη ενός συστήματος συστάσεων. Αυτό το σύστημα θα μπορεί να προτείνει ποιοι υπερσύνδεσμοι (\selectlanguage{english}URLs\selectlanguage{greek}) ενός ιστοτόπου είναι πιο πιθανό να ενδιαφέρουν το χρήστη. Τα δεδομένα που συλλέγονται θα αναλυθούν χρησιμοποιώντας αλγόριθμους μηχανικής μάθησης για τον εντοπισμό μοτίβων και τάσεων στη συμπεριφορά των χρηστών. Στη συνέχεια, αυτές οι πληροφορίες θα χρησιμοποιηθούν για την εκπαίδευση του συστήματος συστάσεων ώστε να κάνει πιο ακριβείς και χρήσιμες προτάσεις.
\section*{Οδηγίες Εγκατάστασης}
\addcontentsline{toc}{section}{Οδηγίες Εγκατάστασης}

Αρχικά πρέπει να κατεβάσετε την επέκταση


Αρχικά θα χρειαστείτε ένα φυλλομετρητή (\selectlanguage{english}Google Chrome\selectlanguage{greek} ή \selectlanguage{english}Firefox\selectlanguage{greek}). Αν δεν έχετε ήδη κάποιον από τους δύο φυλλομετρητές μπορείτε να εγκαταστήσετε το φυλλομετρητή \selectlanguage{english}Google Chrome\selectlanguage{greek} πατώντας \href{https://www.google.com/intl/en_uk/chrome/dr/download/?brand=JJTC&gclid=Cj0KCQjwtsCgBhDEARIsAE7RYh0p-X2d7kNkVFLPkWmugeV2VobmgNNGBRYp2fAA-5cahaJSEYdx50AaAlgwEALw_wcB&gclsrc=aw.ds}{εδώ} και τον \selectlanguage{english}Firefox\selectlanguage{greek} πατώντας \href{https://www.mozilla.org/en-GB/firefox/new/}{εδώ}.

\begin{figure}[H]
    \centering
    \begin{subfigure}{.45\textwidth}
        \includegraphics[width=\textwidth]{logo}
        \caption{\selectlanguage{english}Google Chrome\selectlanguage{greek}}
        \label{Fig:sub1}
    \end{subfigure}
    \hfill
    \begin{subfigure}{.45\textwidth}
        \includegraphics[width=\textwidth]{logo}
        \caption{\selectlanguage{english}Firefox\selectlanguage{greek}}
        \label{Fig:sub2}
    \end{subfigure}
\end{figure}

% todo add explanation how to download the extension
\selectlanguage{english}
\subsection*{Google Chrome}
\addcontentsline{toc}{subsection}{Google Chrome}
\selectlanguage{greek}Αν έχετε το φυλλομετρητή \selectlanguage{english}Google Chrome\selectlanguage{greek}, ανοίξτε το φυλλομετρητή σας και μεταβείτε στη σελίδα των επεκτάσεων. Αυτό μπορείτε να το κάνετε πληκτρολογώντας \selectlanguage{english}chrome://extensions\selectlanguage{greek} στη γραμμή διευθύνσεων ή κάνοντας κλικ στο μενού με τις τρις κουκκίδες στην πάνω δεξιά γωνία και επιλέγοντας \textbf{Περρισσότερα Εργαλεία $\rightarrow$ Επεκτάσεις}. Θα πρέπει να έχετε μπροστά σας την παρακάτω οθόνη.

\begin{figure}[H]
    \includegraphics[width=\textwidth]{logo}
    \caption*{Αρχική οθόνη \selectlanguage{english}chrome://extensions\selectlanguage{greek}}
\end{figure}

Στη συνέχεια πρέπει να ενεργοποιήσετε τη λειτουργία για προγραμματιστές. Ο τρόπος για να γίνει αυτό είναι κανοντας κλικ στον διακόπτη στην πάνω δεξιά γωνία της σελίδας όπως φαίνεται στην παρακάτω εικόνα.
\begin{figure}[H]
    \includegraphics[width=\textwidth]{logo}
    \caption*{Λειτουργία για προγραμματιστές}
\end{figure}

Μετά πρέπει να φορτώσετε την επέκταση στον \selectlanguage{english}Google Chrome\selectlanguage{greek}. Για να γίνει αυτό πρέπει να πατήσετε στο \selectlanguage{english}Load Unpacked\selectlanguage{greek} (το οποίο εμφανίστηκε αφού ενεργοποιήσατε τη λειτουργία για προγραμματιστές) στο πάνω αριστερό μέρος της σελίδας και να επιλέξετε το φάκελο που έχετε την επέκταση.
\begin{figure}[H]
    \centering
    \begin{subfigure}{.45\textwidth}
        \includegraphics[width=\textwidth]{logo}
        \caption{\selectlanguage{english}Load Unpacked\selectlanguage{greek}}
        \label{Fig:sub3}
    \end{subfigure}
    \hfill
    \begin{subfigure}{.45\textwidth}
        \includegraphics[width=\textwidth]{logo}
        \caption{Επιλογή φακέλου}
        \label{Fig:sub4}
    \end{subfigure}
\end{figure}

Η επέκταση θα πρέπει να έχει εκγατασταθεί με επιτυχία και να βλέπετε την παρακάτω οθόνη
\begin{figure}[H]
    \includegraphics[width=\textwidth]{logo}
    \caption*{Επέκταση εγκατεστημένη}
\end{figure}

\selectlanguage{english}
\subsection*{Firefox}
\addcontentsline{toc}{subsection}{Firefox}
\selectlanguage{greek}
\selectlanguage{greek}Αν έχετε το φυλλομετρητή \selectlanguage{english}Firefox\selectlanguage{greek}, ανοίξτε το φυλλομετρητή σας και μεταβείτε στη σελίδα των επεκτάσεων. Αυτό μπορείτε να το κάνετε πληκτρολογώντας \selectlanguage{english}aboud://addons\selectlanguage{greek} στη γραμμή διευθύνσεων ή κάνοντας κλικ στο μενού με τις τρις οριζόντιες γραμμές στην πάνω δεξιά γωνία και επιλέγοντας \textbf{Πρόσθετα και Θέματα}. Θα πρέπει να έχετε μπροστά σας την παρακάτω οθόνη.

\begin{figure}[H]
    \includegraphics[width=\textwidth]{logo}
    \caption*{Διαχείριση πρόσθετων}
\end{figure}

Πατήστε στο γρανάζι (\faCog) όπως φάινεται στην παραπάνω εικόνα και επιλέξτε \textbf{εγκατάσταση πρόσθετου από αρχείο}. Επιλέγετε το αρχείο που κατεβάσατε νωρίτερα με κατάληξη \selectlanguage{english}.xpi\selectlanguage{greek}.

\begin{figure}[H]
    \includegraphics[width=\textwidth]{logo}
    \caption*{Επιλογή αρχείου}
\end{figure}

Ο \selectlanguage{english}Firefox\selectlanguage{greek} θα εμφανίσει ένα προειδοποιητικό μήνυμα το οποίο σας ζητά να επιβεβαιώσετε την εγκατάσταση. Κάντε κλικ στο \textquotedblleftΠροσθήκη\textquotedblright για να προχωρήσετε. Η επέκταση θα έχει πλέον εγκατασταθεί και θα πρέπει να βλέπετε τα παρακάτω.

\begin{figure}[H]
    \includegraphics[width=\textwidth]{logo}
    \caption*{Πρόσθετο εγκατεστημένο}
\end{figure}

\section*{Οδηγίες Χρήσης}
\addcontentsline{toc}{section}{Οδηγίες Χρήσης}


\end{document}

