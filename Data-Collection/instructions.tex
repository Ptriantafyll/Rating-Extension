\documentclass{article}
\usepackage{lipsum}

\usepackage[LGR, T1]{fontenc} %για γλώσσα
\usepackage[greek,english]{babel}
\usepackage[utf8]{inputenc}

\usepackage{graphicx} %για εικόνες
\usepackage{float} 
\usepackage{wrapfig} 
\usepackage{color}   %May be necessary if you want to color links
\usepackage{textcomp}
\usepackage{xcolor}

\usepackage{caption}
\usepackage{hyperref}
\usepackage{tocloft}

% Add dots to table of contents
\renewcommand{\cftsecleader}{\cftdotfill{\cftdotsep}}

\hypersetup{pdfborder=0 0 0}

\begin{document}
\selectlanguage{Greek}
\begin{titlepage}

    \begin{figure}
        \centering
        \includegraphics[scale=0.4]{Upatras}
    \end{figure}

    \begin{center}
        \line(1,0){300}\\
        [0.25in]
        \huge{\bfseries \selectlanguage{greek}Συλλογή δεδομένων για διπλωματική εργασία}\\

        \line(1,0){200}\\
        \begin{figure}[H]
            \centering
            \includegraphics[scale=2.2]{logo}\\
        \end{figure}
    \end{center}

    \begin{flushright}
        \textsc{\\}
        \textsc{\\}

        \textsc{\large \\ Τριανταφυλλόπουλος Παναγιώτης\\ΑΜ 1054367}
        \textsc{\large \\ Τμήμα Μηχ. Η/Υ \& Πληροφορικής}
        \textsc{\large \\ Διπλωματική Εργασία}
    \end{flushright}

\end{titlepage}

\tableofcontents
\thispagestyle{empty}
\cleardoublepage

\setcounter{page}{1}

\selectlanguage{english}
\section*{Disclaimer}
\addcontentsline{toc}{section}{Disclaimer}

\selectlanguage{greek} Αυτό το ερευνητικό έργο θα χρησιμοποιήσει έναν τυχαία δημιουργημένο κωδικό αναγνώρισης για να διασφαλίσει την ανωνυμία όλων των συμμετεχόντων. Δε θα συλλεχθούν ούτε θα αποθηκευθούν προσωπικά στοιχεία ταυτοποίησης και δεν θα γίνει προσπάθεια σύνδεσης οποιουδήποτε συγκεκριμένου συμμετέχοντος με τις απαντήσεις ή τα δεδομένα του. Όλα τα δεδομένα που συλλέγονται κατά τη διάρκεια αυτής της μελέτης θα διατηρηθούν αυστηρά απόρρητα και θα χρησιμοποιηθούν αποκλειστικά για ερευνητικούς σκοπούς. Η συμμετοχή σε αυτή τη μελέτη είναι εντελώς εθελοντική και οι συμμετέχοντες μπορούν να αποσυρθούν ανά πάσα στιγμή χωρίς κυρώσεις ή απώλεια ωφελημάτων που διαφορετικά δικαιούνται.

\section*{Εισαγωγή}
\addcontentsline{toc}{section}{Εισαγωγή}

Καλείστε να συμμετάσχετε στη συλλογή δεδομένων στα πλαίσια της διπλωματικής εργασίας του προπτυχιακού φοιτητή Τριανταφυλλόπουλου Παναγιώτη, η οποία έχει σκοπό τη δημιουργία ενός συστήματος συστάσεων ιστοσελίδων. Για τη συλλογή δεδομένων θα χρειαστείτε έναν φυλλομετρητή (\selectlanguage{english}browser\selectlanguage{greek} - κατά προτίμηση \selectlanguage{english}Google Chrome\selectlanguage{greek} ή \selectlanguage{english}Firefox\selectlanguage{greek}) και να εγκαταστήσετε και να χρησιμοποιήσετε μία επέκταση φυλλομετρητή (\selectlanguage{english}browser extension\selectlanguage{greek}) που έχει δημιουργηθεί από τον φοιτητή.

\section*{Σκοπός συλλογής δεδομένων}
\addcontentsline{toc}{section}{Σκοπός συλλογής δεδομένων}

Ο σκοπός της συλλογής δεδομένων από χρήστες που χρησιμοποιούν την επέκταση φυλλομετρητή είναι η ανάπτυξη ενός συστήματος συστάσεων. Αυτό το σύστημα θα μπορεί να προτείνει ποιοι υπερσύνδεσμοι (\selectlanguage{english}URLs\selectlanguage{greek}) ενός ιστοτόπου είναι πιο πιθανό να ενδιαφέρουν το χρήστη. Τα δεδομένα που συλλέγονται θα αναλυθούν χρησιμοποιώντας αλγόριθμους μηχανικής μάθησης για τον εντοπισμό μοτίβων και τάσεων στη συμπεριφορά των χρηστών. Στη συνέχεια, αυτές οι πληροφορίες θα χρησιμοποιηθούν για την εκπαίδευση του συστήματος συστάσεων ώστε να κάνει πιο ακριβείς και χρήσιμες προτάσεις.
\section*{Οδηγίες Εγκατάστασης}
\addcontentsline{toc}{section}{Οδηγίες Εγκατάστασης}

\section*{Οδιγίες Χρήσης}
\addcontentsline{toc}{section}{Οδηγίες Χρήσης}


\end{document}

